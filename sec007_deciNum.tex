\section{Dezimalzahlen, Dezimalbrüche}\vspace{-1em}
Eine weitere Möglichkeit Zahlen außerhalb der ganzen Zahlen darzustellen sind \emph{Dezimalzahlen} (Kommazahlen). Sie erweitern das uns bekannte Stellwertsystem von Einern, Zehnern, Hundertern usw. um Zehntel, Hunderstel, Tausendstel usw. Die Zahl 1337 setzt sich dabei z.B. zusammen aus $1$ Tausender, $3$ Hundertern, $3$ Zehnern und $7$ Einern.

Die Zahl $0{,}75$ bedeutet dabei, dass es $0$ Einer, $7$ Zehntel und $5$ Hunderstel gibt. Also 
\begin{equation*}
	0{,}75 = 0+\frac{7}{10}+\frac{5}{100}= \frac{70}{100}+\frac{5}{100}=\frac{75}{100}\underset{25}{=}\frac{3}{4}
\end{equation*}
Hinweis: Die Stellenwerte nach dem Komma werden oft als kleine Buchstaben geschrieben und entsprechen gespiegelt denen vor der einer Stelle. Beispiel: $1\, 598\, 186{,}435$:
\begin{center}
	\begin{tabular}{c|c|c|c|c|c|c|c|c|c|c}
		M & HT & ZT & T& H&Z&E&,&z&h&t\\ \hline
		Mio. & 100Tsd. & 10Tsd. & 1000er & 100er & 10er & Einer&, & 10tel & 100stel & 1000stel \\ \hline
		1&5&9&8&1&8&6&,&4&3&5
	\end{tabular}
\end{center}