\section{Addition und Subtraktion von Dezimalzahlen}\vspace{-1em}
Wie schon bei der Addition und Subtraktion von ganzen Zahlen kann addiert werden, wenn die Stellenwerte übereinander stehen. Im Zweifel müssen noch Nullen angehängt werden (Nullen hinter der letzten Zahl hinter dem Komma ändern den Wert der Zahl nicht).

Beispiele Addition:\\
\begin{tabular}{cllll}
	& T   & H   & Z     & E   \\
	&     & $3$ & $5$   & $7$ \\
	$+$ & $2$ & $4$ & $3_1$ & $9$ \\ \hline
	$=$ & $2$ & $7$ & $9$   & $6$
\end{tabular}
%
\hfill
%
\begin{tabular}{cllll}
	& E,   & z   & h     & t   \\
	&    $0$, & $3$ & $5$   & $7$ \\
	$+$ & $2$, & $4$ & $3_1$ & $9$ \\ \hline
	$=$ & $2$, & $7$ & $9$   & $6$
\end{tabular}
%
\hfill
%
\begin{tabular}{clllll}
	& H   & Z   &  E,    & z & h   \\
	& $3$ & $6$ & $5$, & $0$ & $5$ \\
	$+$ & $2_1$ & $4$ & $3$,$_1$ & $9_1$ & $7$ \\ \hline
	$=$ & $6$ & $0$ & $9$,   & $0$ & $2$
\end{tabular}
%

Im ersten und zweiten Beispiel kann man sehen, dass sich außer dem Setzen des Kommas nichts verändert. Die Stellenwerte werden hinter dem Komma fortgesetzt nach den Zehnerpotenzen (10, 100, 1000) als Bruch. Im Dritten Beispiel kann man sehen, dass der Übertrag auch über das Komma hinweg also von der Zehntel zur Einer-Stelle geschieht.

Beispiele Subtraktion:\\
\begin{tabular}{cllll}
	& T     & H   & Z     & E   \\
	& $4$   & $3$ & $6$   & $7$ \\
	$-$ & $2_1$ & $4$ & $3_1$ & $9$ \\ \hline
	$=$ & $1$   & $9$ & $2$   & $8$
\end{tabular}
%
\hfill
%
\begin{tabular}{cllll}
	& E,       & z   & h     & t   \\
	& $4$,     & $3$ & $6$   & $7$ \\
	$-$ & $2$,$_1$ & $4$ & $3_1$ & $9$ \\ \hline
	$=$ & $1$,     & $9$ & $2$   & $8$
\end{tabular}
%
\hfill
%
\begin{tabular}{clllll}
	& H     & Z   & E,       & z     & h   \\
	& $3$   & $6$ & $5$,     & $0$   & $0$ \\
	$-$ & $2$ & $4$ & $3$,$_1$ & $8_1$ & $7$ \\ \hline
	$=$  & $1$   & $2$ & $1$,     & $1$   & $3$
\end{tabular}

Ist eine der beteiligten Zahlen \emph{zu kurz}, können hinter dem Komma beliebig viele Nullen angehängt werden, sodass die Zahlen gleich viele Nachkommastellen besitzen. Das ist zum Beispiel beim letzten Beispiel geschehen, indem an $365$ noch zwei \emph{unsichtbare Nullen} angehängt wurden.