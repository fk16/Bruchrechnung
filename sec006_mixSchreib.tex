\section{Gemischte Schreibweise}\vspace{-1em}
Wir alle kennen den Ausdruck \emph{anderthalb Stunden} für den Zeitraum von \emph{ein einhalb} Stunden. Gemeint ist damit die Summe aus einer Stunde und einer halben Stunde. Mathematisch wird dies abgekürzt geschrieben als $1\frac{1}{2}h$. Diese Schreibweise lässt sich umrechnen in einen \emph{gewöhnlichen} Bruch, indem die ganze Zahl vor dem Bruch mit dem Nenner multipliziert wird und zum Zähler addiert wird.
\begin{equation*}
	1\frac{1}{2}= 1+\frac{1}{2} = \frac{2}{2}+\frac{1}{2}=\frac{3}{2}=\frac{1\cdot 2+1}{2}
\end{equation*}
Ebenso
\begin{equation*}
	3\frac{7}{8}= \frac{3\cdot 8}{8}+\frac{7}{8}=\frac{24+7}{8}= \frac{31}{8}
\end{equation*}
Die Rückrichtung funktioniert so, dass man bei einem Bruch mit $Z\ddot{a}hler>Nenner$ den Zähler so zerlegt, dass ein Summand sich durch den Nenner teilen lässt und der andere kleiner als der Nenner ist:
\begin{equation*}
	\frac{43}{7}=\frac{42+1}{7}=\frac{42}{7}+\frac{1}{7}=6+\frac{1}{7}=6\frac{1}{7}
\end{equation*}
Im Beispiel ist $43>7$ der Zähler größer als der Nenner. Den Zähler zerlegen wir in $42+1$, da $42$ durch $7$ teilbar ist und $1<7$. Dann wird die Summe auf zwei Brüche mit Nenner $7$ zerlegt und der erste Summand zur ganzen Zahl $6$ gekürzt. Schließlich wird das Plus weggelassen und wir erhalten einen Bruch in gemischter Schreibweise.
Ein weiteres Beispiel:
\begin{equation*}
	\frac{77}{6}=\frac{72+5}{6}=\frac{72}{6}+\frac{5}{6}=12+\frac{5}{6}=12\frac{5}{6}
\end{equation*}
Im Zähler wird die $77$ zerlegt in $72+5$, weil $72=6\cdot12$ ist und $5<6$.

Vorteil: Man muss nur noch den Bruch kürzen, die Zahl vor dem Bruch bleibt immer gleich.

\subsubsection*{Übungen:}\vspace{-1em}
Überführe in die jeweils andere Schreibweise. Kürze ggfs. das Ergebnis.
\begin{multicols}{2}
	\begin{enumerate}[label=\alph*)]
		\item $\frac{8}{6}$
		\item $\frac{65}{12}$
		\item $3\frac{5}{6}$
		\item $5\frac{3}{4}$
	\end{enumerate}
\end{multicols}
\ RED{Lösung für Übung gemischte Schreibweise}