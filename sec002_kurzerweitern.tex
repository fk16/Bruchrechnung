\section{Kürzen und Erweitern}\vspace{-1em}
Vielleicht ist es dir bei der letzten Übung schon aufgefallen: Manche Bruchteile geben den gleichen Anteil an, ohne, dass sie gleich aussehen. Es ist egal, ob ich $\frac{6}{8}$ von einem Kilometer oder $\frac{3}{4}$ von einem Kilometer berechne. Das Ergebnis sind immer $750m$. Das liegt daran, dass beide Brüche den gleichen Wert haben. Man kann also schreiben
\begin{equation*}
	\frac{6}{8}=\frac{3}{4}
\end{equation*}
Führen wir uns die dreiviertel Stunde von weiter oben nochmal vor Augen, kann man auch sagen, dass $\frac{45}{60}=\frac{3}{4}$ ist. Du kannst die letzte Gleichung lesen als \glqq{}$45$ von $60$ sind das Gleiche wie $3$ von $4$\grqq{} oder auch als \glqq{}$45$ sechzigstel sind gleich $3$ viertel\grqq{}.

Der Wechsel zwischen verschiedenen Schreibweisen von Brüchen mit dem gleichen Wert geschieht durch \emph{kürzen und erweitern}. Dabei werden Zähler und Nenner eines Bruches mit der gleichen Zahl multipliziert (erweitern) oder durch die gleiche Zahl dividiert (kürzen). Beim Kürzen muss die \emph{Kürzungszahl} ein gemeinsamer Teiler von Zähler und Nenner sein.

Im Fall von $\frac{45}{60}$ bringt uns eine Kürzung um $15$ auf den Bruch $\frac{3}{4}$, weil 
\begin{equation*}
	\frac{45}{60}=\frac{45:15}{60:15}=\frac{3}{4}
\end{equation*}

Die Gegenrichtung (erweitern) verläuft so
\begin{equation*}
	\frac{3}{4}= \frac{3\cdot 15}{4\cdot 15}=\frac{45}{60}
\end{equation*}

Auch die eben angesprochenen $\frac{6}{8}$ lassen sich durch Kürzen um $2$ auf die Form $\frac{3}{4}$ bringen:
\begin{equation*}
	\frac{6}{8}=\frac{6:2}{8:2}=\frac{3}{4}
\end{equation*}

Eine Kürzung oder Erweiterung wird manchmal auch abgekürzt geschrieben mit Zahlen über und unter dem Gleichheitszeichen:
\begin{itemize}[]
	\item \glqq{}Erweitere $\frac{5}{7}$ um $3$\grqq{} wird dann zu $\frac{5}{7}\overset{3}{=}\frac{15}{21}$. \hfill (Nebenrechnung  $5\cdot3=15$;  $7\cdot 3 = 21$)
	\item \glqq{}Kürze $\frac{75}{100}$ mit $25$\grqq{} wird dann zu $\frac{75}{100} \underset{25}{=}\frac{3}{4}$. \hfill (Nebenrechnung $75:25=3$;  $100:25=4$)
\end{itemize}

Kürzen und Erweitern kann in mehreren Schritten geschehen. Mit dem größten gemeinsamen Teiler kommt man durch Kürzen immer sofort auf die Darstellung mit dem kleinsten Zähler und Nenner. Diese Darstellung wird \emph{vollständig gekürzt} genannt. Die vollständig gekürzte Darstellung von $\frac{45}{60}$ ist zum Beispiel $\frac{3}{4}$. Denn $15$ ist der größte gemeinsame Teiler von $45$ und $60$ und oben haben wir bereits gezeigt, dass $\frac{45}{60}\underset{15}{=}\frac{3}{4}$ ist.

\COLBOX{
	Merke: Beim Erweitern werden Zähler und Nenner mit der gleichen Zahl multipliziert. Beim Kürzen werden Zähler und Nenner durch der gleichen Zahl dividiert.
	\\[1em]
	Beim Erweitern und Kürzen ändert sich der Wert des Bruches nicht. Die dargestellten Anteile werden nur verfeinert (erweitern) bzw. zusammengefasst (kürzen).
	\\[1em]
	Bei einem vollständig gekürzten Bruch sind Zähler und Nenner \emph{teilerfremd}. Es gibt keine gemeinsamen Teiler.
}

\subsubsection*{Übungen:}\vspace{-1em}
\begin{enumerate}[label=\alph*)]
	\item Erweitere $\frac{11}{12}$ um $4$.
	\item Erweitere $\frac{5}{6}$ um $10$.
	\item Kürze $\frac{12}{60}$ mit $6$.
	\item Kürze $\frac{48}{60}$ soweit wie möglich (vollständig gekürzt).
\end{enumerate}
Die Lösungen sind in der Fußnote\footnote{a) $\frac{44}{48}$; b) $\frac{50}{60}$; c) $\frac{2}{10}$; d) $\frac{4}{5}$}