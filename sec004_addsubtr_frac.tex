	\section{Addition und Subtraktion von Brüchen}\vspace{-1em}

\COLBOX{
	Brüche werden addiert bzw. subtrahiert, indem sie zunächst auf den gleichen Nenner gebracht werden (\emph{gleichnamig machen}) und anschließend die Zähler entweder summiert werden oder die Differenz der Zähler gebildet wird.
}
Hinweis: Die Addition bzw. Subtraktion geschieht bei gleichnamigen Brüchen ausschließlich im Zähler. Der Nenner bleibt gleich.

Beispiel Addition:
\begin{equation*}
	\frac{2}{3}+\frac{1}{4}= \frac{8}{12}+\frac{3}{12}= \frac{8+3}{12}=\frac{11}{12}
\end{equation*}

Beispiel Subtraktion:
\begin{equation*}
	\frac{6}{9}-\frac{2}{6}= \frac{2}{3}-\frac{1}{3}= \frac{2-1}{3}=\frac{1}{3}
\end{equation*}
Bei der Subtraktion wurde hier gekürzt. Man hätte ebenso auf den gemeinsamen Nenner $18$ erweitern können. In diesem Fall wäre das Ergebnis $\frac{6}{18}$ gewesen. Das ist (gekürzt um $6$) gleich $\frac{1}{3}$. Kürzen und Erweitern können in jeder Kombination genutzt werden.

\subsubsection*{Übungen:}\vspace{-1em}
\begin{enumerate}[label=\alph*)]
	\item \todo{Übung Add subtr frac}
\end{enumerate}