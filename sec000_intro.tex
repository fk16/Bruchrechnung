	\section{Einleitung}\vspace{-1em}
Wir alle kennen Brüche aus unserem Alltag. Jeder von uns hat schon von einer \emph{viertel Stunde} gehört und selbst ein \emph{halbes Pfund} könnte einigen geläufig sein. Mit diesen Angaben geben wir Größen an, die kleiner als ein Ganzes sind. Die \emph{vier}tel Stunde ist der \emph{vier}te Teil einer Stunde bzw. $60:4 = 15$ Minuten. Das halbe Pfund ist die Hälfte von $500g$ also $500g:2=250g$.

Neben diesen im Alltag sehr verbreiteten Brüchen lassen sich auch andere Bruchteile angeben:
\begin{itemize}[noitemsep]
	\item Eine Stunde ist der vierundzwanzigste Teil eines Tages. \hfill $\frac{1}{24}d=1h$
	\item Ein Zentimeter ist der hundertste Teil eines Meters. \hfill$\frac{1}{100}m=1cm$
	\item Eine Minute ist der sechzigste Teil einer Stunde. \hfill $\frac{1}{60}h=1min.$
\end{itemize}
Du kannst dir vorstellen, dass es im Grunde für alle Größen auch kleinere \emph{Anteilsgrößen} gibt. Die erste Anwendung von Brüchen ist daher das Berechnen von Anteilen von Größen bzw. Bruchteilen von Größen. Die Zahl unter dem Bruchstrich gibt an, in wie viele Teile das Ganze zerlegt wurde. Brüche werden nach dieser Zahl benannt:
\begin{center}
	\begin{tabular}{c|c|c|c|c|c|c|c|c|c}
		Bruch & $\frac{1}{2}$ & $\frac{1}{3}$ & $\frac{1}{4}$ & $\frac{1}{5}$ & $\frac{1}{6}$ & $\frac{1}{7}$ & $\frac{1}{8}$ & ... & $\frac{1}{20}$ \\ \hline
		Name  & halbe              & drittel & viertel & fünftel & sechstel & siebtel & achtel &... & zwanzigstel
	\end{tabular}
\end{center}
Ab der Zahl vier wird an die Zahl die Endsilbe \glqq{}-tel\grqq{} oder bei Zahlwortendung auf \glqq{}-g\grqq{} (wie bei zwanzig) die Endsilbe \glqq{}-stel\grqq{} angehängt. Bei der Zahl sieben wir dauf das \glqq{}-en\grqq{} am Ende verzichtet.\footnote{Im Englischen werden Bruchteile als \emph{half}, \emph{third}, \emph{fourth}, \emph{fifth} usw. benannt. Hier ist ab der Zahl vier die Endung immer \glqq-th\grqq{}. Bei Zahlen, die auf \glqq{}-y\grqq{} enden, wie z.B. \emph{twenty}, wird das \glqq{}-y\grqq{} durch \glqq{}-ieth\grqq{} ersetzt. So erhält man z.B. das \emph{twentieth} von \emph{20th century Fox}.}