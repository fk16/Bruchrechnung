\section{Bruchteile von Größen\label{sec:bruchteile}}\vspace{-1em}
Eben hatten wir bereits eine viertel Stunde angesprochen. Aber auch eine drei viertel Stunde kennst du vermutlich schon. In der Mathematik schreiben wir die drei viertel Stunde so: $\frac{3}{4}h$. 

Ist der Bruchteil einer Größe gefordert, so kann man die Aufgabe lösen, in dem man zunächst (1) die Größe in die nächst kleinere Größe umwandelt und dann (2) durch den Nenner teilt. Abschließend wird (3) mit dem Zähler multipliziert.

Wir schauen uns diese drei Schritte mal am Beispiel der dreiviertel Stunde an:
\begin{enumerate}
	\item Gesucht ist $\frac{3}{4}$ von $1h$. $1h=60min$. \hfill $\frac{3}{4}$ von $60min$.
	\item Der Nenner ist $4$ daher teile $60$ durch $4$. \hfill $60:4=15$
	\item Der Zähler ist $3$ multipliziere $15$ mit $3$. \hfill $15\cdot 3 = 45$
\end{enumerate}

Diese Rechnung lässt sich auch am Stück abhandeln. Wir erhalten dann eine Kette von Rechnungen:
\begin{equation*}
	\frac{3}{4}\text{ von }1h=\frac{3}{4}\cdot 1h \overset{(1)}{=} \frac{3}{4}\cdot 60min = 60min : 4 \cdot 3 \overset{(2)}{=} 15 min \cdot 3 \overset{(3)}{=} 45 min
\end{equation*}
Bei der Kettenrechnung dürfen die Division und Multiplikation vertauscht werden. 
Auf die gleiche Weise lassen sich auch andere Bruchteile berechnen.

\subsubsection*{Übungen:}\vspace{-1em}
Prüfe durch eine Rechnung:
\begin{enumerate}[label=\alph*)]
	\item Drei viertel von $1m$ sind $75cm$.
	\item Fünf sechstel von $1h$ sind $50min$.
	\item Sechs achtel von $1km$ sind $750m$.
	\item Drei achtel von $4km$ sind $1500m$.
\end{enumerate}
