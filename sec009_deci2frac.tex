\section{Umwandlung von Dezimalzahl zu Bruch}\vspace{-1em}
\subsection{Abbrechende Dezimalzahl}\vspace{-1em}
Eine abbrechende Dezimalzahl kann in einen Bruch umgewandelt werden, indem man zunächst die Zahl vor dem Komma als Ganzes lässt und dann die Zahlen hinter dem Komma als Zähler auf einen Bruch schreibt. Der Nenner ist diejenige Zehnerpotenz (zehn, hundert, tausend,...), an der die letzte Zahl steht.

Beispiel:
\begin{equation*}
	27{,}3654=27+\frac{3\,654}{10\,000}= \frac{273\,654}{10\,000}=27\frac{3\,654}{10\,000}=27\frac{1\,827}{5\,000}
\end{equation*}
Die Anzahl der Nullen hinter der $1$ im Nenner entspricht der Position der letzten Ziffer der Dezimalzahl hinter dem Komma.

\subsection{Periodische Dezimalzahl}\vspace{-1em}
Bei der Umwandlung von $\frac{1}{9}$, $\frac{2}{9}$, $\frac{3}{9}$,... in eine Dezimalzahl fällt auf, dass der Zähler über dem Nenner $9$ genau die Zahl in der Periode ist:
$\frac{1}{9}=0{,}\overline{1}$, $\frac{2}{9}=0{,}\overline{2}$, $\frac{3}{9}=0{,}\overline{3}$. Einstellige Perioden sind daher immer genau die Zahl in der Periode über dem Nenner $9$.

... schwierig als Text ... \RED{Periodische Brüche in DeziZahlen umwandeln erklären}